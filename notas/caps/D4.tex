\section{Teórico}

  \definicion{Topic:} Advanced functionals: higher accuracy (hybrids), strongly correlated materials (DFT+U), weakly bound systems (van der Waals).

  \definicion{Speaker:} Stefano DE GIRONCOLI (SISSA, Italy).

\subsection{Funcionales en DFT: término XC}

  Existen diferencias generaciones o grupos de funcionales que van creciendo en precisión, pero esto conlleva un incremento en su complejidad (Fig. \ref{fig:jacob}): incluso el funcional más sencillo contiene información acerca de la densidad de carga del sistema.

  \begin{figure}[H]
      \centering
      \includegraphics[scale = 0.5]{figs/D4/Jacob.png}
      \caption{Escalera de Jacob de las aproximaciones de DFT.}
      \label{fig:jacob}
  \end{figure}

  Los teoremas de Hohenberg-Kohn (HK) están relacionados a cualquier sistema que consista de electrones moviéndose bajo la influencia de un potencial externo:
    \begin{enumerate}
      \item \textbf{Primer teorema HK:} el potencial externo y, en consecuencia, la energía total son un funcional único de la densidad electrónica. Luego, la densidad electrónica del estado fundamental describe de manera unívoca el potencial y, por lo tanto, todas las propiedades del sistema, incluyendo la función de onda many-body. Se sigue que el espectro del Hamiltoniano también es un funcional único de dicha densidad basal. Particularmente, el funcional HK es una función universal de la densidad electrónica, no dependiendo de manera explícita del potencial externo.
        $$F[n] = T[n] + U[n]$$
      \item \textbf{Segundo teorema HK:} El funcional permite determinar la energía del estado fundamental si y sólo si la densidad electrónica es la del estado fundamental.
        $$F [n] = \underbrace{min}_{\Psi \rightarrow n} \bra{\Psi} T_e + W_{ee} \ket{\Psi}$$
    \end{enumerate}

  Se tiene entonces que para cualquier natural $N$ y potencial externo $v_{ext} (\vec{r})$ existe un funcional de desidad $F[n]$ tal que
    $$E [n] = F[n] + \int v_{ext} (\vec{r}) n (\vec{r}) d\vec{r}$$

  Partiendo de esta idea, KS plantean la energía cinética de un sistema ficticio de electrones no interactuantes.
    $$T_s[n] = \underbrace{min}_{\Psi \rightarrow n} \bra{\Psi} T_e\ket{\Psi}$$

  Con esta información se reescribe la ecuación de HK según del sistema interactuante según
    $$F[n] = T_s[n] + E_{Har} [n] + E_{XC} [n]$$

  Así la energía del sistema será
    $$E [n] = T_s[n] + E_{Har} [n] + E_{XC} [n] + \int v_{ext} (\vec{r}) n (\vec{r}) d\vec{r}$$

  Todo lo que quedó fuera al recurrir a este sistema no interactuante está contenido en el término XC. Este término es pequeño relativo a los demás términos que contribuyen a la energía total del sistema. Por este motivo es que incluso la aproximación más sencilla sigue siendo útil para los cálculos.

\subsection{L(S)DA}

  La aproximación más sencilla es Local (Spin) Density Approximation (L(S)DA), donde la "S" se pone según se esté considerando o no la contribución de spin. En este caso basta con conocer la densidad electrónica en cada punto del espacio.
    $$E_{XC} = E_{XC} [n]$$

  Asume que la densidad varía lentamente, tratando entonces a la densidad local como un gas homogéneo de electrones: el valor de la energía XC en una posición $\vec{r}$ se determina exclusivamente a partir del valor de la densidad en esa posición, \emph{i.e} el valor local de $n$.

  A pesar de su sencillez, ofrece resultados bastante razonables para el estado fundamental. Describe una gran variedad de propiedades para un amplio abanico de materailes: energía, estabilidad de fases, defectos termodinámicos, geometrías de equilibrio, funciones respuesta, etcétera. Si, además, consideramos la aproximación adiabática, la dinámica de la red es una propiedad del estado fundamental: puedo determinar propiedades vibracionales, termidonámicas y defectos. En general se observan buenas tendencias respecto a los valores experimentales. También da buenas tendencias.

  Las primeras limitantes son que da lugar a enlaces más cortos y más fuertes, dando en consecuencia módulos de Young mayores. Además los bandgaps son muy chicos.

  \Obs{Cuánto mayor sea tasa de cambio de la densidad, menos confiables son los resultados cuando se usa LDA.}

\subsection{GGA}

  Esta familia de funcionales conocido como Generalized Gradient Approximation (GGA) requiere información del gradiente local de la densidad electrónica además de la propia densidad electrónica.
    $$E_{XC} = E_{XC} [n, \nabla n]$$

  En realidad se utiliza el gradiente reducido, dado por
    $$\frac{\nabla n}{n^{\sfrac{4}{3}}}$$

  La densidad decae exponencialmente, por lo que denominador provoca que este cociente sea divergente para $r$ grande. Esto hace que en la región de mayor probabilidad de encontrar electrones (en torno al núcleo), el gradiente sea pequeño y acotado. Sin embargo, al alejarnos del núcleo la densidad cae y el gradiente reducido diverge. Se tiene entonces que el funcional puede considerar dos regiones: una más externa o superficial y una más interna o cercana al núcleo. Esto permite una mucho mejor descripción del sistema.

  El primero de este tipo fue PW91 (Perdew-Wang). Basándose en éste, simplificando las ideas, pero haciéndolas a su vez más potentes en sus resultados, surgió uno de los más usados: PBE (Perdew-Burke-Ernzerhof).

  Estos resuelven el overbinding del LDA. Se acercan al valor real, pero siguen estando por encima en algunos casos. Aunque sus efectos en los parámetros de red son más aleatorios. De todas formas logra describir mucho mejor las estructuras cristalinas de los elementos. Además, GGA es importante para sistemas magnéticos, particularmente para el hierro: LDA dice que es no magnético, pero GGA logra describir que tiene propiedades magnéticas.

  Los problemas que había con LDA y que GGA no logra superar son:
    \begin{itemize}
      \item Buenas tendencias para enlaces fuertes (covalente, iónico, metálico), pero no para pequeñas superposiciones.
      \item Las autoitneracciones no son 100 \% canceladas. Cuanto más localización, la autointeracción se vuelve más relevante. También en sistemas sólidos que conservan mucho sus cualidades atómicas (orbitaes moleculares con apariencia de orbitales atómicos).
      \item Como son tan locales, no logran ver cosas no localizadas como las vdW. Cuando se llega a un resutlado acorde suele ser más por errores numéricos.
    \end{itemize}

\subsection{Solución a la autointeracción}

  Para resolver el problema de las autointeracciones se recurre a SIC (Self Interaction Correction), DFT+U o funcionales híbridos.

  En principio, en DFT los electrones interactúan con un potencial efectivo generado por todos los electrones, incluido él mismo. Cuando la densidad es más dispersa, el error de autointeracción es pequeño. Sin embargo, cuanto más localizado se vuelve, mayor es error. El problema también se presenta cuando cambia la localidad del orbital, por ejemplo en una redox que transfieren un electron desde un estado deslocalizado a uno localizado. Esto es usual en óxidos de metales de transición, generando errores del orden de los eV.

  El problema de la autointeracción viene de la mano del tratamiento que se le hace a la aproximación de XC en términos de electrostática: LDA es bueno describiendo el movimiento de un electrón en un potencial medio, resultando mejor cuanto más lejos estén los electrones entre sí.

  El funcional SIC no es muy utilizado, pero su desarrollo fue conceptualmente importante. Es una solución ad hoc orbital-dependent, pero complicada.

  Los funcionales híbridos combinan SIC con LDA/GGA. Es muy caro de aplicar con una base de ondas planas. DFT+U son funcionales útiles para describir materiales fuertemente correlacionados. Recientemente ha sido aplicado a materiales más generales con resultados prometedores.

\subsection{Conexión adiabática}

  Dijimos que HK considera un Hamiltoniano con todas las interacciones \emph{prendidas} mientras que KS describe un sistema no interactuante. Podemos pensar que ambos funcionales son dos casos extremos de una misma situación. La conexión la podemos hacer introduciendo un parámetro $\lambda \in[0,1]$ KS (0) hasta el sistema many-body de HK (1). El parámetro varía de forma continua y adiabática.

  Utilizando el teorema de Hellmann-Feynman, se llega a que la suma entre la energía de Hartree y la XC es el promedio desde la no interacción hasta interaccion full sobre el valor de expectación de la matriz de interacción.

  Becke asume que la variacion de la perturbación con $\lambda$ cambia suave y linealmente. Para $\lambda=0$ es el XC de Hartree computado con los orbitales KS, mientras que para $\lambda=1$ se requieren otras aproximaciones como LDA. Esto es lo que se conoce como el funcional half-half (HH). De acá se pueden ver B3LYP, PBEO y HSE.

\subsection{Energía de Hartree-Fock}

  El término HF es muy caro de calcular (10 a 100 veces más) con PW ya que va y vuelve entre espacios duales y en cada uno hace cálculos caros para cada orbital.

  Cuando escribimos la integral HF en forma recíproca, tiene una divergencia en $q+G=0$. Este no es un problema porque es una singularidad evitable, pero de manera no trivial: remueven el término divergente y luego agregan un término que contiene esta divergencia atenuada con una exponencial.

\subsection{DFT+U}

  La idea de hacer DFT+U es tratar la fuerte interacción Coulómbica de los electrones localizados (la cual no está bien desripta con LDA ni GGA) agregando un término Hubbard. Esto es relevante principalmente en metales donde las interacciones Coulómbicas son particularmente fuertes en los orbitales d y f.
   $$E_{DFT+U} = E_{DFT} + E_{U} \Rightarrow E_{U} = \frac{U}{2} \sum_a tr\left[ \rho_a (1 - \rho_a)  \right]$$

  donde $\rho_a$ es la matriz de ocupación de OA. En otros términos estamos agregando una penalización a la energía DFT para forzar la matriz de ocupación a alcanzar la idempotencia (o completamente ocupada o compeltamente desocupada). Las ocupaciones fraccionarias no son compatibles con valores altos de U: a mayor U, mayor penalty.

  \Obs{Para tener una idea gráfica, podemos mapear una parábola invertida donde $U$ es su máximo: pushea al sistema a caer en ocuáncias 0 ó 1 ya que las fracciones tienen un precio, siendo el $U$ el más caro cuando la ocupación es $\sfrac{1}{2}$. El término de Hubbard implica agregar un potencial de Hubbard a la ecuacion KS. La corrección introduce una discontinuidad en el término XC al ser un corrimiento rígido del potencial.}

  Calcular $U$ en sólidos no es tan sencillo ya que uno no tiene tanto control sobre cuántos electrones se disponen en cada obrital: esto lo decide KS. Así que hay que sacarlo de la curvatura de $E_{DFT}$ con respecto al número de ocupación. En la práctica se introducen perturbaciones locales sobre grandes superceldas y luego se calculan las variaciones en la energía con respecto a los números de ocupación. Al invertir la función respuesta, sale $U$. Para hacer este cálculo se tiene el ejecutable hp.x (Hubbard Parameters) en QE. No es conveniente para close-shell systems.

  Dijimos que la corrección se aplica sobre orbitales altamente localizados, generalmente d y f. Para saber sobre qué orbitales aplicar la corrección, conviene graficar la PDOS para ver dónde andan los orbitales d y/o f y si están parcialmente ocupados o no. Generalmente se utiliza sobre metales de transición, lantánidos y actínidos.

  \Obs{DFT+U suele ser peor que los funcionales híbridos.}

  También tenemos DFT+U+V (FIg. \ref{fig:DFT+U+V}): una fomulación extendida que considera una interacción de Hubbard entre sitios interatómicos $V$ además de la contribución $U$ que depende de un único sitio. Este funcional extendido tiene más flexibilidad: no sólo arregla el problema de localización, sino que también corrige la interacción con ligandos. El valor de V también se puede conocer con hp.x.

  \begin{figure}[H]
      \centering
      \includegraphics[scale = 0.6]{figs/D4/DFT+U+V.png}
      \caption{DFT+U+V.}
      \label{fig:DFT+U+V}
  \end{figure}

\subsection{Funcionales de vdW}

  Se trata de un efecto de correlación no local incluido en XC, pero que no está considerado en ningún funcional: vdW necesita considerar al menos 2 puntos.

  Hay varias opciones:
    \begin{itemize}
      \item Despreciarlo.
      \item Agregar una correción de dispersión amortigüada empírica (Grimme).
      \item Desarrollar un funcional XC no local.
    \end{itemize}

\subsection{Aspectos generales}

  \begin{itemize}
    \item \textbf{L(S)DA:} simples y bien definidos. Dan una buena geometría, pero overbinding.
    \item \textbf{GGA:} gran variedad para elegir, con mejoras en los resultados de energía. Buena geometría.
    \item \textbf{Meta-GGA:} muy complicados y no tan usados.
    \item \textbf{SIC, DFT+U, Híbridos:} atacan el problema de la autointeracción.
    \item \textbf{vdW:} falta mucho. No corrigen las autointeracciones: en tal caso deben usarse conjuntamente vdW y DFT+U.
  \end{itemize}

\section{Q\&A}

  \definicion{In which cases do we need to use the pseudopotential with relativistic corrections?}

  Tipically when your system contains heavy elements, tipically from the 4th period of the periodic table on.

  For those elements the relativistic corrections can be important. Just for the sake of clarity: the scalar relativistic effects (mass and Darwin correction) are always present in the scalar relativistic pseudopotentials. The full relativistic correction (spin-orbit, basically) can be added with the lspinorb flag in the input file and with an appropriate fully relativistic pseudopotential. The spin-orbit correction starts to become somehow relevant from the 4th period of the periodic table on.

  \definicion{What is the difference between k and q meshes?}

  The k mesh is the mesh of k points for the electronic calculation. The q mesh, instead is a mesh of q points for the perturbation theory calculation. More details on q meshes will be given tomorrow, in the lecture and the hands-on session. Quick reply, q is the wave vector of the perturbation you apply to the system.

  q-meshes have a different meaning in hybrids and in DFPT to compute U/phonons. In all cases one has to converge the result of interest with respect to the density of the q-mesh

  \definicion{Is it advisable to adjust the exx\_fraction prameter for hybrid calculation depending on the system studied or the functional used? For example, the default value is 0.25 for input\_dft ='PBE0'. Can one change this parameter or set it by performing a convergence test w.r.t the experimental band gap of any system, e.g., a molecule?}

  I would say NO... the functional parametrization (B3LYP, PBE0, etc...) is the result of extensive balancing and tuning procedures. if you change it to fit the band gap it is very likely that you are messing up with other properties. so I would say this option is for very advanced users and for cases where you exactly know what you are doing

  \definicion{Is symmetry used for the q-mesh for hybrids, i.e., selecting a set of 'irreducible q-points'?}

  actually no... in the sense that the k-mesh for the first BZ summation is reduced using symmetry, the q-mesh for the, possibly coarser grid, is complete...

  Unfortunately you cannot use symmetry twice for the double sum over k and q!

  \definicion{How could I change the hp.FeO.in if want to compute also V?}

  For the moment we do not have a tutorial for DFT+U+V, In your case, you need to have this:
      lda\_plus\_u = .true.,
      lda\_plus\_u\_kind = 2,
      U\_projection\_type = 'atomic',
      Hubbard\_V(1,1,1) = 1.d-8
      Hubbard\_V(2,2,1) = 1.d-8
      Hubbard\_V(3,3,1) = 1.d-8
  Then use the HP code to compute U and V - see example here: QE/HP/examples/example10

\section{Hands-on}

\definicion{Topic:} Functionals.

\definicion{Speaker:} Iurii TIMROV (EPFL, Switzerland).

\subsection{Objetivo}

  Utilizar funcionales avanzados (Hubbard U, hybrids, and Van der Waals) para modelado de materiales.

\subsection{DFT+U}

  Queremos calcular la PDOS de FeO usando DFT y DFT+U para comparar resultados.

  Para determinar sobre qué orbitales debemos apLicar la corrección, debemos recurrir a una parte de código duro de QE (aún no implementado vía input): $set\_hubbard\_n.f90$ y $set\_hubbard\_l.f90$ donde $n$ y $l$ refiere respectivamente a los números cuánticos principal y orbital. Queremos aplicar la $U$ sobre los electrones 3d ($n=3$ y $l=2$).

  Para activar la maquinaria DFT+U hay que poner $lda\_plus\_u = .true.$. Las versiones de DFT+U son: $lda\_plus\_u\_kind = 0,1,2$, donde $0$ es la versión más simple, $1$ es más completa y $2$ es DFT+U+V. Luego, $U\_projection\_type = 'atomic'$ le dice qué tipo de funciones considerar para la matriz de ocupación. En este caso son funciones tipo atómicas. Podría ser $'orto'$ y serían atómicas, pero mútuamente ortogonales, incluyendo también los oxígenos. Esto arroja resultados más precisos. Finalmente, tenemos $Hubbard\_U(i) = 1.d-8$ donde $i$ es el índice del átomo sobre el cual aplicar la corrección. Son casi cero para hacer un cálculo PBESol (PBE sobre sólidos), pero no cero para activar la maquinaria DFT+U. Esto generará información extra en el output que nos va a servir (LDA+U parameters al final: spin 1 = up y spin 2 = down). Con esto tan bajito no vamos a hacer ninguna correción U, sólo es para pedirle que imprima la matriz de ocupación. Como son hierros cistalográficamente equivalentes, deben tener el mismo valor en $Hubbard\_U(i)$.

  Primero hacemos la DFT estándar:
    \begin{verbatim}
      pw.x < pw.FeO.scf.in > pw.FeO.scf.out
      pw.x < pw.FeO.nscf.in > pw.FeO.nscf.out
      projwfc.x < projwfc.FeO.in > projwfc.FeO.out
      gnuplot plot\_pdos.gp
    \end{verbatim}

  La nscf es para tener la PDOS. Con $nbnd = 35$ agrega más antiestados para que la cosa salga linda. Vemos que para el átomo 1 el spin up (spin 1) está prácticamente lleno, pero spin down (spin 2) tiene ocupaciones parciales (0.2). En el segundo hierro tenemos las cosas invertidas.
    \begin{figure}[H]
        \centering
        \includegraphics[scale = 0.5]{figs/D4/LAD+U.png}
    \end{figure}

  Esto se refleja en la PDOS: la majority está toda por debajo de la Fermi y para la minority tenemos un poco abajo y otro poco arriba de la energía de Fermi. Vemos entonces que al hacer DFT estándar el sistema es metálico.

  \begin{figure}[H]
      \centering
      \includegraphics[scale = 0.5]{figs/D4/FeO_pre.png}
  \end{figure}

  Ahora vamos a calcular el valor de la U de Hubbard, partiendo desde el valor casi nulo que le hemos asignado. Luego
    \begin{verbatim}
      pw.x < pw.FeO.scf.in > pw.FeO.scf.out
      hp.x < hp.FeO.in > hp.FeO.out
    \end{verbatim}

    \begin{figure}[H]
        \centering
        \includegraphics[scale = 0.5]{figs/D4/U.png}
    \end{figure}

  Ahora usamos el valor obtenido para hacer una corrida DFT+U: modificamos los inputs $pw.FeO.scf.in$ y $pw.FeO.nscf.in$ cambiando los valores de la U de Hubbard a 4.6 (eV). En este caso estos valores hacen referencia a los estados 3d de ambos hierros. Luego hacer los mismos pasos que antes.

  Vemos que cuando hacemos la corrección Hubbard ahora tenemos que el sistema es un aislante ya que se nos presenta el bandgap, lo cual es experimentalmente correcto.

\subsection{DFT+hybrids}

  Vamos a calcular la energía total del Si usando DFT con el funcional híbrido PBE0.

  Los funcionales híbridos son realmente caros. Con $input\_dft$ le decimos qué funcional híbrido usar (pbe0, b3lyp, hse). Para esto, debemos usar un PP que sea lo más cercano al híbrido que queremos. Si queremos que se encargue de la singularidad, usamos $x\_gamma\_extrapolation = .true.$. Le pedimos que la integre analíticamente diciendo $exxdive\_treatment = 'gygi-baldereschi'$.

  \Obs{nqx1 = 1, nqx2 = 1, nqx3 = 1 es la q-mesh, que en este caso sería $\Gamma$.}

  Primero vamos a hacer el test de convergencia usando $x\_gamma\_extrapolation = .true.$ con q-mesh 1x1x1, 2x2x2, 4x4x4, and 8x8x8. Luego hacemos el test de convergencia de la energía total usando $x\_gamma\_extrapolation = .false.$ y los mismos q-mesh.
  \begin{verbatim}
    pw.x < pw.Si.scf.in > pw.Si.scf.out
  \end{verbatim}

  No hice todo porque es muy lento, pero el resultado sería como la figura a continuación. El cálculo con $x\_gamma\_extrapolation = .true.$ es muchísimomás rápido. Además, usar extrapolación permite usar una $q$-mesh más chiquita.

  \begin{figure}[H]
      \centering
      \includegraphics[scale = 0.5]{figs/D4/hibridos.png}
  \end{figure}

\subsection{DFT+vdW}

  Vamos a estudiar grafito con DFT usando funcionales de vdW. La idea es ver la interacción de vdW entre layers de grafeno. Primero debemos hacer una vc-relax.
  \begin{verbatim}
    pw.x < pw.graphite.vc-relax.in > pw.graphite.vc-relax.out
  \end{verbatim}

  Luego estudiar los siguientes casos para comparar la distancia entre capas (valor experimental: 3.336 A):
  \begin{itemize}
    \item Sin considerar vdW:
      \begin{itemize}
        \item Normal LDA calculation   @LDA pseudo
        \item Normal PBE calculation   @PBE pseudo
      \end{itemize}
    \item Considerando vdW de manera no local:
      \begin{itemize}
        \item `input_dft = 'vdw-DF'`    @PBE pseudo (non-local)
        \item `input_dft = 'vdw-DF2'`   @PBE pseudo (non-local)
        \item `input_dft = 'rVV10'`     @PBE pseudo (non-local)
      \end{itemize}
    \item Considerando vdW de manera semiempírico:
      \begin{itemize}
        \item `vdw_corr  = 'DFT-D'`     @PBE pseudo (semi-empirical)
        \item `vdw_corr  = 'DFT-D3'`    @PBE pseudo (semi-empirical)
      \end{itemize}
  \end{itemize}

  \Obs{Mientras que $input\_dft$ cambia el funcional (cambia desde el comienzo la movida tropical), $vdw\_corr$ hace correcciones al final del cálculo (tipo PP).}

  \begin{figure}[H]
      \centering
      \includegraphics[scale = 0.5]{figs/D4/vdW.png}
  \end{figure}

  La distancia de equilibrio entre layers es muy pequeña con LDA y muy larga con GGA. Esto indica que debemos considerar vdW. ¿Cuál es la mejor? Habría que hacer test de convergencia primero. 
