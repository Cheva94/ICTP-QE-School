\section{Teórico}

  \definicion{Topic:} Introduction to Magnetism.

  \definicion{Speaker:}	Alessandro STROPPA (CNR-SPIN, Italy).

\subsection{Integrales de Coulomb y de intercambio}

  19

  20 - figura que resume todo

  2J -- diferencia energética entre el singlete y el triple.

  Interacción Coulómbica interelectrónica junto al postulado de simetrización da lugar al intercambio.

  Partiendo de un Hamiltoniano no spin llegamos al spin.

  26 -- de vuelta lo del Hamiltoniao, pero ahora con dos opciones para J (pos o neg).

\subsection{Hamiltoniao de Heisenberg}

  27 -- A = -2J

  de Heisenberg = de intercambio

  ver la 28 -- timpos de interacciones magnética

\subsection{Interacción spin-órbita}

  29 -- 10-100 veces menor que el término de intercambio, pero muy importante para el magnetismo.

  30 -- contruye la energía magnética y el momento magnético. Termina llegando a acople LS.

  El splitting puede ir dsde 5x10-5 hasta [...]

  31 -- perspectiva relativista

  32 -- si el momento orbital prefiere estar en una cierta dirección ..

\subsection{High vs low spin}

  36 -- Hund + Pauli

\subsection{Campo de intercambio - interacci Zeeman}

  37

\subsection{Conceptos claves para magnetismo en sólidos}

  38

  39 -- localización de cargas (ver items)

\subsection{Potencial centrífugo: magnetismo localizado vs itinerante}

  40 --

  41 -- hay algo que aclara la 40

  42, 43 -- figuras interesantes

\subsection{Modelo de bandas del ferromagnetismo}

  45 -- El más simple: modelo de Stoner. ferromagnetismo metálico

  Algo de smearing y de DFT

  46 -- modelo de Stoner para la banda 3d

\subsection{Summary}

  59 y 60 hace una lista

\subsection{Spin DFT}

  66 -- extendieron el sistema KS a sistemas spin-polarizados, introduciendo una matriz 2x2 para la densidad electrónicas.

  ver la disgresión (67 y 68 y 69)

  llega a una matriz densidad descompuesta en CL de matrices de Pauli.

  72 -- usa esta descomposición para expresar el potencial XC magnético. -- 73

  Ecuación a resolver -- 73 (ver comentarios)

\subsection{Magnetismo colinear y no colinear}

  75 - no colinear es el caso más general: no existe un único eje de cuantización. En el caso colinear el eje de cuantización es único y se llama z.

  76 -- no colinear y colineal (matriz 100pre diagonal). Tengo densidad de carga spin up y spin down.

  77 -- evaluaciones

  78 -- densidad de carga y de spin.

\subsection{Summary}

  79 --

  80 -- ver comentarios

  83 -- dibujitos

\subsection{DFT}

  81 -- qué podemos obtener

\section{Q\&A}

\section{Hands-on}

  \definicion{Topic:} Introduction to Magnetism.

  \definicion{Speaker:}	Pietro DELUGAS (SISSA, Italy).
